\chapter{SPI Slave}

The SPI slave is an active peripheral in the sense that it receives/sends data
without the assistance of the core. In fact is intended as an external interface
through which a user of the system can access the internal memories from
outside.
This can be used to pre-load programs into the memories, start the system, wait
for an acknowledgment that the program has terminated and then examine the
results.

The SPI slave has an AXI master through which it can access all peripherals and
memories. From the outside it can be accesses by sending SPI/QSPI commands to it.
QSPI is an extension to SPI that uses four data lanes instead of one and is thus
four times as fast as the standard SPI.

If pins on an ASIC are scarce, it is possible to just connect the standard SPI
interface to pins and accept the performance loss in transfers from/to the SoC.
