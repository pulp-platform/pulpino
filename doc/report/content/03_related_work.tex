%%%%%%%%%%%%%%%%%%%%%%%%%%%%%%%%%%%%%%%%%%%%%%%%%%%%%%%%%%%%%%%%%%%%%%%
%%%%%%%%%%%%%%%%%%%%%%%%%%%%%%%%%%%%%%%%%%%%%%%%%%%%%%%%%%%%%%%%%%%%%%%
%%%%%                                                                 %
%%%%%     <file_name>.tex                                             %
%%%%%                                                                 %
%%%%% Author:      <author>                                           %
%%%%% Created:     <date>                                             %
%%%%% Description: <description>                                      %
%%%%%                                                                 %
%%%%%%%%%%%%%%%%%%%%%%%%%%%%%%%%%%%%%%%%%%%%%%%%%%%%%%%%%%%%%%%%%%%%%%%
%%%%%%%%%%%%%%%%%%%%%%%%%%%%%%%%%%%%%%%%%%%%%%%%%%%%%%%%%%%%%%%%%%%%%%%


\chapter{Related Work}

At the moment one can see a landsliding transition happening in the open-source hardware community. This began with the 
effort of the OpenRISC project in (?) which was the first open-source release of a micro-controller like architecture. 
At the moment this is currently climaxing with Berkely's RISCV project. 

\section{OpenRISC}


\section{RISC-V}

RISC-V is a project initiated by the electrical engineering department of University of California Berkely (UCB). It aims
to create an open and freely available Instruction Set Architecture (ISA) standard. The design of the ISA aims satisfy a very broad 
field of application purposes. Ranging from small scale micro-controllers to full-blown out-of-order many core architectures.

Specifically interesting in relation to the present work is their Z-scale implementation. It features a 32-bit 3-stage single-issue in-order 
pipeline with support for the RV32IM ISA (integer base arithmetics and multiplication) and M/U privilege modes. Communication with
the memories takes place over a 32bit AHB-lite bus.

Currently UCB provides to versions of the z-scale core. One is written in their own hardware description language (HDL) called chisel 
while the other implementation is conducted entirely in Verilog. Both are distributed under a 3-clause BSD license.

The emerging ecosystem that comes along with the growing popularity of the RISC-V ISA comes in handy for the \pulpino project as well. 
The various virtual platforms provided by UCB can emulate code that will natively run on \pulpino. In addition it provides us with 
compilers that at least support the official RISC-V ISA.

\section{lowRISC}
