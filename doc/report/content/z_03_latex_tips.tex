%%%%%%%%%%%%%%%%%%%%%%%%%%%%%%%%%%%%%%%%%%%%%%%%%%%%%%%%%%%%%%%%%%%%%%%
%%%%%%%%%%%%%%%%%%%%%%%%%%%%%%%%%%%%%%%%%%%%%%%%%%%%%%%%%%%%%%%%%%%%%%%
%%%%%                                                                 %
%%%%%     z_03_latex_tips.tex                                         %
%%%%%                                                                 %
%%%%% Author:      Michael Muehlberghuber (<mbgh@iis.ee.ethz.ch>      %
%%%%% Created:     01.07.2012                                         %
%%%%% Description: Some LaTeX-specific writing tips                   %
%%%%%                                                                 %
%%%%%                                                                 %
%%%%% History:                                                        %
%%%%%%%%%%%%%%                                                        %
%%%%%                                                                 %
%%%%% 01-Jul-2012 (Michael Muehlberghuber - mbgh@iis.ee.ethz.ch):     %
%%%%% *) Created initial version.                                     %
%%%%%                                                                 %
%%%%%%%%%%%%%%%%%%%%%%%%%%%%%%%%%%%%%%%%%%%%%%%%%%%%%%%%%%%%%%%%%%%%%%%
%%%%%%%%%%%%%%%%%%%%%%%%%%%%%%%%%%%%%%%%%%%%%%%%%%%%%%%%%%%%%%%%%%%%%%%

\chapter{\LaTeX{} Tips}

Writing a report with \LaTeX{} may not be as intuitive as it is the
case with \gls{wysiwyg} editors. Especially if you are using \LaTeX{}
(more or less) the first time, some problems with the syntax will
occur. In general, the present document should already serve as a good
starting point for your report and in the best case you only have to
insert the content of your project based on this framework.

Nevertheless, I will try to give some useful tips with regard to
\LaTeX{} throughout the next sections, which may help you to increase
the quality of your documents even further. If you want to use any of
the presented ideas, simply copy the \LaTeX{} source code of the
appropriate section to your on document and adapt it accordingly.


\section{Compiling a \LaTeX{} Document}
\label{sec:compiling}

Basically, either \shell{latex} or \shell{pdflatex} can be used in
order to generate the document output in \gls{dvi} or \gls{pdf}
format, respectively. Throughout this section I will solely use the
\shell{pdflatex} command for demonstration purposes (if you prefer a
DVI document, just replace the \shell{pdflatex} command by
\shell{latex}0).

Compiling a latex document at the \gls{iis} computers is, in general,
as simple as executing the following command in a UNIX terminal
window:

\begin{shellenv}
pdflatex <document_name>
\end{shellenv}

Currently\footnote{State: July 2012} a \TeX{} Live version from the
year 2008 is the default distribution at the \gls{iis}. In order to
use the present \LaTeX{} framework for your report, you have to use a
more up-to-date version of \TeX{} Live, because the framework uses
some \LaTeX{} packages which are not part of the 2008 version. I
suggest using the 2011 version of \TeX{} Live. The simplest way to
check that you can build the report template successfully, is by
executing:

\begin{shellenv}
pdflatex-2011 report_template.tex
\end{shellenv}

This should (re)generate the \gls{pdf} output of the report template,
i.e., the file you are currently reading through. If typing in the
\shell{-2011} postfix becomes annoying for you, you may add aliases
into your \file{.cshrc} as follows:

\begin{shellenv}
alias latex 'latex-2011'
alias pdflatex 'pdflatex-2011'
\end{shellenv}

If you also want to (re)build the glossaries (maybe you have added
some acronyms or the like), you have to compile your report together
with the glossaries as follows:

\begin{shellenv}
pdflatex-2011 your_report.tex
makeglossaries-2011 your_report
pdflatex-2011 your_report.tex
\end{shellenv}

Furthermore, when you modify the references of your report (within the
bibliography file), you also have to (re)run \textsc{Bib}\TeX{} in
order to update your bibliography, i.e.:

\begin{shellenv}
pdflatex-2011 your_report.tex
bibtex-2011 your_report
pdflatex-2011 your_report.tex
pdflatex-2011 your_report.tex
\end{shellenv}


\section{Figures}
\label{sec:figures}

In order to include an image into your report (as it has been done
within in the previous sample chapters), you may use the
\texttt{figure} floating environment. With that, \LaTeX{} will take
care of placing them nicely and you can focus on the actual content of
your document. Figure~\ref{fig:std_eth_logo} shows an example of how
to insert a single figure.

\begin{figure}[htbp]
 \centering\includegraphics[width=0.4\linewidth]{./figures/eth_logo}
 \caption{Standard ETH logo.}
 \label{fig:std_eth_logo}
\end{figure}

If you want to place multiple figures side-by-side, you can do this
with the use of \texttt{minipages}. Figure~\ref{fig:left_eth_logo} and
\ref{fig:right_eth_logo} illustrates an example.

\begin{figure}[htbp]
 \begin{minipage}[t]{0.45\linewidth}
  \centering\includegraphics[width=1.0\linewidth]{./figures/eth_logo}
  \caption{Left ETH logo.}
  \label{fig:left_eth_logo}
 \end{minipage}\hfill
 \begin{minipage}[t]{0.45\linewidth}
  \centering\includegraphics[width=1.0\linewidth]{./figures/eth_logo}
  \caption{Right ETH logo.}
  \label{fig:right_eth_logo}
 \end{minipage}
\end{figure}

In order to create a single figure with multiple subfigures, you can
do this as presented in Figure~\ref{fig:eth_logo_sub}

\begin{figure}[htbp]
 \centering
 \subfloat[Left ETH logo.]{\includegraphics[width=0.3\textwidth]{./figures/eth_logo}}\hfill
 \subfloat[Center ETH logo.]{\includegraphics[width=0.3\textwidth]{./figures/eth_logo}}\hfill
 \subfloat[Right ETH logo.]{\includegraphics[width=0.3\textwidth]{./figures/eth_logo}}
 \caption{Multiple ETH logos as subfigures.}
 \label{fig:eth_logo_sub}
\end{figure}


\section{Tables}
\label{sec:tables}

Tables in \LaTeX{} allow you to present your results quite
nicely. Table~\ref{tab:std_table} shows a standard table.

\begin{table}[htbp]
 \caption{Standard table.}
 \label{tab:std_table}
 \centering\begin{tabular}{@{}lcr@{}} \toprule
  \textbf{Row 1 - Column 1} & \textbf{Row 1 - Column 2} & \textbf{Row 1 - Column 3} \\ \midrule
  Row 2 - Column 1 & Row 2 - Column 2 & Row 2 - Column 3 \\
  Row 3 - Column 1 & Row 3 - Column 2 & Row 3 - Column 3 \\
  Row 4 - Column 1 & Row 4 - Column 2 & Row 4 - Column 3 \\ \bottomrule
 \end{tabular}
\end{table}

Sometimes you may want to add a table which stretches one of its
columns in order to reach the full width of the document. Such an
example is shown in Table~\ref{tab:stretched_table}.

\begin{table}[htbp]
 \caption{Stretched table.}
 \label{tab:stretched_table}
 \centering\begin{tabularx}{1.0\linewidth}{@{}Xcr@{}} \toprule
  \textbf{Row 1 - Column 1} & \textbf{Row 1 - Column 2} & \textbf{Row 1 - Column 3} \\ \midrule
  Row 2 - Column 1 & Row 2 - Column 2 & Row 2 - Column 3 \\
  Row 3 - Column 1 & Row 3 - Column 2 & Row 3 - Column 3 \\
  Row 4 - Column 1 & Row 4 - Column 2 & Row 4 - Column 3 \\ \bottomrule
 \end{tabularx}
\end{table}

If you need to place two tables next to each other, you may use an
approach based on \texttt{minipages} as shown in Table~\ref{tbl:left}
and Table~\ref{tbl:right}.

\begin{figure}
  \begin{minipage}{0.49\textwidth}
    \captionof{table}{Left table.}
    \label{tbl:left}
    \centering\begin{tabular}{@{}lr@{}} \toprule
      Row 1 - Column 1 & Row 1 - Column 2 \\ \midrule
      Row 2 - Column 1 & Row 2 - Column 2 \\
      Row 3 - Column 1 & Row 3 - Column 2 \\
      Row 4 - Column 1 & Row 4 - Column 2 \\ \bottomrule
    \end{tabular}
  \end{minipage} \hfill
  %
  \begin{minipage}{0.49\textwidth}
    \captionof{table}{Right table.}
    \label{tbl:right}
    \centering\begin{tabular}{@{}lr@{}} \toprule
      Row 1 - Column 1 & Row 1 - Column 2 \\ \midrule
      Row 2 - Column 1 & Row 2 - Column 2 \\
      Row 3 - Column 1 & Row 3 - Column 2 \\
      Row 4 - Column 1 & Row 4 - Column 2 \\ \bottomrule
    \end{tabular}
  \end{minipage}
\end{figure}

\section{Creating Glossaries}

In order to generate a glossary within your report (e.g., a list of
acronyms or an actual glossary), take a look into the file
\texttt{glossaries.tex}. There, you will find some examples on how to
define an acronym as well as a glossary entry. If you want to
reference one of the acronyms within your report, you can do it the
same way as I did it with the \gls{led} right here (just take a look
into the source code).

As already mentioned in Section~\ref{sec:compiling}, you have to
rebuild your glossaries in order to display changes. For that, you
first have to build your document using \texttt{latex-2011} or
\texttt{pdflatex-2011} in a shell window, or the \textit{build-button}
in your preferred \LaTeX{} editor GUI. Next, you have to call
\texttt{makeglossaries-2011 \parameter{file\_name}} in a shell
window\footnote{The \texttt{makeglossaries} script is a Perl script
  available at the IIS computer system and should also be part of most
  \TeX{} distributions.}, followed by another build process of your
main source file, i.e.:

\begin{shellenv}
pdflatex-2011 your_report.tex
makeglossaries-2011 your_report
pdflatex-2011 your_report.tex
\end{shellenv}


\section{Creating Algorithm Boxes}

Algorithm boxes in \LaTeX{} allow you to present your algorithms in
pseudo code as shown in the following example:

\begin{algorithm}
  \SetKwData{Left}{left}\SetKwData{This}{this}\SetKwData{Up}{up}
  \SetKwFunction{Union}{Union}\SetKwFunction{FindCompress}{FindCompress}
  \SetKwInOut{Input}{input}\SetKwInOut{Output}{output}

  \Input{A bitmap $Im$ of size $w\times l$}
  \Output{A partition of the bitmap}
  \BlankLine
  \emph{special treatment of the first line}\;
  \For{$i\leftarrow 2$ \KwTo $l$}{
    \emph{special treatment of the first element of line $i$}\;
    \For{$j\leftarrow 2$ \KwTo $w$}{\label{forins}
      \Left$\leftarrow$ \FindCompress{$Im[i,j-1]$}\;
      \Up$\leftarrow$ \FindCompress{$Im[i-1,]$}\;
      \This$\leftarrow$ \FindCompress{$Im[i,j]$}\;
      \If(\tcp*[h]{O(\Left,\This)==1}){\Left compatible with \This}{\label{lt}
        \lIf{\Left $<$ \This}{\Union{\Left,\This}}\;
        \lElse{\Union{\This,\Left}\;}
      }
      \If(\tcp*[f]{O(\Up,\This)==1}){\Up compatible with \This}{\label{ut}
        \lIf{\Up $<$ \This}{\Union{\Up,\This}}\;
        \tcp{\This is put under \Up to keep tree as flat as possible}\label{cmt}
        \lElse{\Union{\This,\Up}}\tcp*[r]{\This linked to \Up}\label{lelse}
      }
    }
    \lForEach{element $e$ of the line $i$}{\FindCompress{p}}
  }
  \caption{disjoint decomposition}\label{algo_disjdecomp}
\end{algorithm}


% \section{Citing}


% \section{Units}

%%% Local Variables:
%%% mode: latex
%%% TeX-master: "../report_template"
%%% End:
