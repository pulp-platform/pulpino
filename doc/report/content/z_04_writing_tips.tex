%%%%%%%%%%%%%%%%%%%%%%%%%%%%%%%%%%%%%%%%%%%%%%%%%%%%%%%%%%%%%%%%%%%%%%%
%%%%%%%%%%%%%%%%%%%%%%%%%%%%%%%%%%%%%%%%%%%%%%%%%%%%%%%%%%%%%%%%%%%%%%%
%%%%%                                                                 %
%%%%%     z_04_writing_tips.tex                                       %
%%%%%                                                                 %
%%%%% Author:      Michael Muehlberghuber (<mbgh@iis.ee.ethz.ch>      %
%%%%% Created:     01.07.2012                                         %
%%%%% Description: Some Writing-specific tips in general.             %
%%%%%                                                                 %
%%%%% History:                                                        %
%%%%%%%%%%%%%%                                                        %
%%%%%                                                                 %
%%%%% 01-Jul-2012 (Michael Muehlberghuber - mbgh@iis.ee.ethz.ch):     %
%%%%% *) Created initial version.                                     %
%%%%%                                                                 %
%%%%%%%%%%%%%%%%%%%%%%%%%%%%%%%%%%%%%%%%%%%%%%%%%%%%%%%%%%%%%%%%%%%%%%%
%%%%%%%%%%%%%%%%%%%%%%%%%%%%%%%%%%%%%%%%%%%%%%%%%%%%%%%%%%%%%%%%%%%%%%%

\chapter{General Writing Guidelines}

As soon as you get familiar with the syntax of \LaTeX{} (and I can
promise you, you will get familiar with it quite quickly as soon as
you start writing your reports with \LaTeX{}), some more general
writing tips might become of interest for your. Therefore, I collected
a few general writing guidlines in the following sections, some of
them with regard to \LaTeX{}, some of them not.

\paragraph{Placement of Floating Environments}
Figures and tables are the two most prominent examples for floating
environments. Although the figure examples presented in
Section~\ref{sec:figures} use \texttt{[htbp]} to tell \LaTeX{} how to
place them, you should normally only use the \texttt{h} parameter if
you really require it. Since \LaTeX{} then at first tries to place the
figure at the same position as its source code, this somehow
contradicts with the actual purpose of the \texttt{figure}
environment. So, in general, try to place floating environments using
one of the following parameters:

\begin{description}
\item[t] Place the floating environment on \textbf{t}op of a page.
\item[b] Place the floating environment on the \textbf{b}ottom of a
  page.
\item[p] Puts the floating environment on a single \textit{floating
    \textbf{p}age} with other floating environments.
\end{description}


\paragraph{Positioning of Figure and Table Captions}
Captions of figures are, in general, placed below the actual figure,
whereas captions of tables should be placed on top of
them. Section~\ref{sec:figures} and \ref{sec:tables} contain some
examples for figures and tables, including correct placement of
captions.


\paragraph{Avoid Unneccessary \LaTeX{} Packages}
Although there are so many ``cool'' \LaTeX{} packages available
everywhere on the Internet, try to use only those, which you really
require. The main problem with loading too many, more or less unknown,
packages is that some of them might redefine some commands, etc.,
which are used by another package which asumes that command to be the
original one. Keeping track of these changes and the relations between
different packages, is quite annoying and takes quite a lot of
time. Hence, keep your preamble simple with regard to packages.


\paragraph{Make Use of Vector Drawings}
Since \LaTeX{} handles vector drawings pretty good and their
scalability allows you to print them in any resolution, prefer them
compared to their pixel counterparts and use them whenever possible.

% \paragraph{Equations Within the ``Text-Flow''}

% \paragraph{Punctuation within Captions}
% \begin{itemize}
%  \item No second fullstop when ending sentences with abbreviations like
%  etc.
% \end{itemize}

%%% Local Variables:
%%% mode: latex
%%% TeX-master: "../report_template"
%%% End:
